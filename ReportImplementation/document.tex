\documentclass[12pt]{article}
\usepackage{graphicx}
\usepackage[utf8]{inputenc}
\usepackage[greek, english]{babel}
\usepackage{alphabeta}
\usepackage{libertine}
%\usepackage{fancyhdr}
\usepackage[hidelinks]{hyperref}  % For hyperlinks without borders

\usepackage{geometry}

\geometry{
	left=0.5in,   % left margin
	right=0.5in,  % right margin
	top=0.6in,    % top margin
	bottom=0.6in  % bottom margin
}

\begin{document}
	
	\begin{titlepage}
		\centering
		\includegraphics[width=0.8\textwidth]{aueb_logo.jpg}
		\vfill
		\Huge\textbf{Your Title}
		\vspace{1cm}
		\Large\textbf{Your Subtitle}
		\vfill
		\today
	\end{titlepage}
	
	\renewcommand{\contentsname}{Περιεχόμενα}
	\tableofcontents
	
	\newpage  % Start content on a new page
	
	\section{Εισαγωγή}
	Το Youtube είναι ένας ιστότοπος κοινοποίησης, αποθήκευσης, αναζήτησης και αναπαραγωγής βίντεο. Κάθε χρήστης μπορεί να δημιουργήσει λογαριασμό και να ανεβάζει τα δικά του βίντεο ή ακόμα και να αναπαράγει σε πραγματικό χρόνο. Εκτός από τους χρήστες, πρόσβαση έχει ο οποιοσδήποτε στον ιστότοπο αυτό όπου μπορεί μόνο να παρακολουθεί τα βίντεο άλλων χρηστών. Το προφίλ του χρήστη παρουσιάζεται ως κανάλι όπου άλλοι χρήστες μπορούν να εγγραφούν ώστε να παρακολουθούν και να ενημερώνονται για βίντεο ή για πραγματικού χρόνου αναπαραγωγές που τους ενδιαφέρουν. Στη παρούσα ανάλυση, καλούμαστε να αναλύσουμε το κανάλι Discovery. Οι τομείς γύρω από τους οποίους στρέφεται το κανάλι αυτό είναι η επιστήμη, η φυσική ιστορία, η ανθρωπολογία, η επιβίωση η γεωγραφία και η μηχανική.
	\label{chap:intro}  % Add a label to refer to this section
	
	\section{Λήψη δεδομένων}
	Τα δεδομενα για αυτη την ανάλυση τα πήραμε με τη χρήση του Bernhard Reiner's Tool.
	Please visit \href{https://www.example.com}{this website}.
	\label{chap:data_fetching}  % Add a label to refer to this section
	
\end{document}
