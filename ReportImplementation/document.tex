\documentclass[12pt]{article}
\usepackage{graphicx}
\usepackage[utf8]{inputenc}
\usepackage[greek, english]{babel}
\usepackage{alphabeta}
\usepackage{libertine}
%\usepackage{fancyhdr}
\usepackage[hidelinks]{hyperref}  % For hyperlinks without borders
\usepackage{hyperref}

\usepackage{geometry}

\usepackage{xcolor}

\usepackage{setspace}
\setstretch{1.5}  % Adjust the line spacing factor

\usepackage{titlesec}
\usepackage{titletoc}
% Adjust the space above and below \section titles
\titleformat{\section}{\normalfont\Large\bfseries}{\thesection}{1em}{}[\vspace{1ex}]
% Adjust the space between sections in the table of contents
\titlecontents{section}[0em]{\vspace{1ex}}{\thecontentslabel\hspace{1em}}{}{\titlerule*[1pc]{.}\contentspage}

\usepackage{caption}

\usepackage{wrapfig}

\usepackage{adjustbox}

\usepackage{amsmath}


\geometry{
	left=0.5in,   % left margin
	right=0.5in,  % right margin
	top=0.6in,    % top margin
	bottom=0.6in  % bottom margin
}

\hypersetup{
	colorlinks=true,
	linkcolor=black,  % set the color for internal links
	urlcolor=blue    % set the color for external links
}

\DeclareCaptionLabelFormat{customlabel}{Εικόνα 3.1}
\captionsetup[figure]{labelformat=customlabel, labelsep=period}

\begin{document}
	
	\begin{titlepage}
		\centering
		\includegraphics[width=0.5\textwidth]{photos-files/aueb_logo.jpg}
		\vfill
		\Huge\textbf{THKEVASE KSANA EKFONISI GIA TO TI ZHTA}
		\vspace{1cm}
		\Large\textbf{Your Subtitle}
		\vfill
		\today
	\end{titlepage}
	
	\renewcommand{\contentsname}{Περιεχόμενα}
	\tableofcontents
	
	\newpage  % Start content on a new page
	
	\section{Εισαγωγή}
	Το Youtube είναι ένας ιστότοπος κοινοποίησης, αποθήκευσης, αναζήτησης και αναπαραγωγής βίντεο. Κάθε χρήστης μπορεί να δημιουργήσει λογαριασμό και να ανεβάζει τα δικά του βίντεο ή ακόμα και να αναπαράγει σε πραγματικό χρόνο. Εκτός από τους χρήστες, πρόσβαση έχει ο οποιοσδήποτε στον ιστότοπο αυτό όπου μπορεί μόνο να παρακολουθεί τα βίντεο άλλων χρηστών. Το προφίλ του χρήστη παρουσιάζεται ως κανάλι όπου άλλοι χρήστες μπορούν να εγγραφούν ώστε να παρακολουθούν και να ενημερώνονται για βίντεο ή για πραγματικού χρόνου αναπαραγωγές που τους ενδιαφέρουν. Τα βίντεο που ανεβάζει ο κάθε χρήστης είναι συνηθως αποθηκευμένα σε playlists αναλόγως με την μορφή και το θέμα που έχουν. Επίσης στο κανάλι του ο κάθε χρήστης μπορεί να έχει κανάλια άλλων χρηστών που όπως αναφέρονται στην αγγλική ορολογία "Featured channels". Τα επιλεγμένα αυτα κανάλια αποτελούν κανάλια όπου ενας χρήστης επιλέγει να τα συμπεριλάβει στο δικο του κανάλι(δεν φαίνονται στο κοινό). Ο λόγος που γίνεται αυτό είναι για να προωθούν οι χρήστες και να εμφανίζουν άλλα κανάλια που τους αρέσουν, με τα οποία μπορεί να συνεργάζονται ή να θέλουν να τα προτείνους στους θεατές τους. Έτσι με αυτό τον τρόπο, οι χρήστες μπορούν να προσεγγίσουν πολλά είδη κοινού και να αυξήσουν ετσι τις εγγραφές και τις προβολές τους. Στην ανάλυση αυτή θα εξετάσουμε το κανάλι Samsung. Το κανάλι αυτό είναι το κανάλι του ομίλου εταιρειών Samsung που έχει ως σκοπό την ενημέρωση σχετικά με εκδηλώσεις, καινοτομες ταιχνολογίες, αφαρμογές και υπηρεσίες,  B2B solutions, παρουσιάσεις, και τις τελευταίες και καινοτόμες τεχνολογίες του ομίλου.
	\label{chap:intro_1}
	
	\section{Λήψη δεδομένων}
	Τα δεδομενα για την ανάλυση μας τα πήραμε με τη χρήση του \href{https://labs.polsys.net}{Bernhard Reiner's Tool} χρησιμοποιόντας τα YouTube Data Tools. Αρχικά, χρησιμοποιόντας το link του καναλιου στο YouTube, βρήκαμε το id του καναλιού μέσω του  \href{https://ytdt.digitalmethods.net/mod_channel_info.php}{Channel Info Module}. Έπειταμ με τη χρήση του \href{https://ytdt.digitalmethods.net/mod_channel_info.php}{Channel Network Module}, πήραμε δεδομένα για το δίκτυο του καναλιού. Οι παραμέτροι που χρησιμοποιήθηκαν ηταν το seed(αρχικό κανάλι) με τη χρηση του id με crawl depth ίσον με 2(το crawl depth καθορίζει πόσο βαθιά στο δίκτυο μπορουμε να φτάσουμε. Για παράδειγμα με depth=0 το εργαλείο αυτο επιστρέφει το δίκτυο με τις συσχετίσεις ανάμεσα στα seeds που δίνονται, με dept=1 επιστρέφει τα featured channels που έχει ο χρήστης στο κανάλι του και με depth=2 επιστρέφει τα featured channels που υπάρχουν στα κανάλια που βρήκαμε στο depth=1). Η επιλογή για της εγγραφές δεν λήφθηκε υπόψην δίοτι θέλαμε τα δεδομένα να είναι μόνο με τα featured channels. Μετά απο αυτά τα βήματα το εργαλείο δημιούργησς ενα gdf αρχείο το οποίο φορτώσαμε στο πρόγραμμα Gephi για ανάλυση. Εδω να σημεωθει οτι μέσω του Gephi έγινε έλεγχος των δεδομένων για τυχόν σφάλατα που θα μπορούσαν να επηρεάσουν την ανάλυση μας όπως για παρέδειγμα ο έλεγχος δυπλοτύπων, όπου σε μια περίπτωση υπήρκε διπλότυπο όπου και εντιμετωπίστηκε μέσω του Gephi, ο έλεγχος για null τιμές κ.α. \textcolor{red}{Σε μερικές περιπτώσεις υπήρχαν μη διαθέσιμες τιμές. Για παράδειγμα σε ορισμένους κόμβους, δεν υπήρχε στο αντίστοιχο κελί η χώρα ενώ ήταν γνωστή. Επομένως εισήχθησαν χοιροκίνητα οι τιμές όπου ηταν εφικτό. Σε άλλες περιπτώσεις, τυχόν σφάλματα αντιμετωπίζονται αναλόγως τη δεδομένη στιγμή όπου και αναφέρονται.}
	\label{chap:data_fetching_2}
	
	
	\section{Γραφική Αναπαράσταση Δικτύου}
	Το δίκτυο μόνο με τα ονόματα των κόμβων(καναλιών) χωρίς κάποια παραμετροποίηση.
		\begin{center}
			\includegraphics[width=0.7\textwidth]{photos-files/section3/first_painting.jpg}
		\end{center}
	\newpage
	
	Επίσης μέσω του Gephi μπορούμε να θέσουμε διάφορες παραμέτρους όσον αφορά τoν χρωμaτισμό και την διάταξη ανάλογα με ορισμένες ιδιότητες που έχει το δίκτυο μας. Για παράδειγμα, για τη μορφή των κόμβων θέσαμε το μέγεθος του κάθε κόμβου ανάλογα με το πλήθος των εγγραφών(subscribercount) που έχει το κανάλι που αντιπροσοπεύει και για τον χρωματισμό θέσαμε ασπρο-πορτοκαλι-κοκκινο στο χαρακτηριστικο των προβολών(viewcount(100s)). Για την διάταξη, τρέξαμε τον Atlas Force 2 για να αραιώσουμε τον γράφο μας και τον Label Adjust για να διαχωριστουν οι ετικέτες ονομάτων των κόμβων. Έτσι προέκυψε η πάρακάτω εικόνα:
		\begin{center}
			\includegraphics[width=0.8\textwidth]{photos-files/section3/second_painting.jpg}
		\end{center}
	Απο την εικόνα αυτή, τα δεδομένα που λαμβάνουμε είνα ο αριθμός των εγγραφών σε ενα κανάλι παίζει αρκετό ρόλο με τις προβολές που μπορεί να εχει, πραγμα αναμενόμενο για τον ιστότοπο που συζητάμε.
	\newpage
	
	Βλέποντας τα δεδομένα του δικτύου μας απο το Data Laboratory του Gephi, παρατηρήσαμε πως υπάρχουν κανάλια απο διάφορες χώρες. Επομένως θεωρήσαμε ενδιαφέρον να κάνουμε μία παραμετροποίηση με τις χώρες ως εξής. Ο χρωματισμός έγινε μέσω διφορετικών χρωματων, τοσων, όσος και ο αριθμός των διαφορετικών χωρών, μέσω του partition tab. Στο σημείο αυτο, θεώρήσαμε επίσης σημαντικό και την αναφορά του seed. Αυτό έγινε μεσω του μεγέθους των κόμβων μέσω του seedrank(αντίστοιχη μεταβλητη με την isseed εαν χρησιμοποιούσαμε τον χρωματισμό). Στη συνέχεια μέσω του Plugin Circular Layout που κατεβάσαμε μέσω των Tools του Gephi, δημιουργήσαμε την πιο κάτω διάταξη θέτοντας στην ιδιότητα "Order Nodes By" την χώρα. Για άλλη μια φορά, χρησιμοποιήσαμε τον Label Adjust για διαχωρισμό των ετοικετών.
		\begin{center}
			\includegraphics[width=0.8\textwidth]{photos-files/section3/third_painting.jpg}
		\end{center}
	Απο την πιο πάνω εικόνα μπορούμε εύκολα να παρατηρήσουμε πως ο κεντρικός και ισως ο πιο σηαντικος κόμβος να είναι ο "Samsung" ο οποίος είναι με πράσινο χρώμα. Οι δύο δεξιές θέσεις απο αυτο το κόμβο είναι επίσης με πράσινο χρώμα αφού και αυτοι οι κόμβοι είναι κανάλια απο την ίδια χώρα, την Νότιο Κορέα.
	\label{chap:graphical_representation_3}
	
	
	\section{Βασικά στοιχεία Δικτύου}
	Το δίκτυο που μελετάμε έχει τα εξής βασικά στοιχεία:
	\begin{itemize}
		\item Αριθμός κόμβων: \textbf{76} διαφορετικά \textbf{κανάλια-κόμβοι}
		\item Αριθμός ακμών: \textbf{149 συνδέσμοι} μέσω των οποίων συνδέονται τα κανάλια-κόμβοι
		\item Ο γράφος μας είναι \textbf{κατευθυνόμενος}. Δηλαδή κάθε σύνδεσμος απο ενα κανάλι προς ενα άλλο εχει κατέυθυνση όπως φαινεται στην πιο κάτω εικόνα:
		\begin{center}
			\includegraphics[width=0.8\textwidth]{photos-files/section4/ate.JPG}
			
			Ο πράσινος κόμβος-κανάλι έχει ως featured channel τον κόμβο-κανάλι με ροζ χρώμα.
		\end{center}
		\item Διάμετρος δικτύου: Η \textbf{διάμετρος} ενός δικτύου είναι η μακρύτερη συντομότερη διαδρομή που μπορούμε να βρούμε. Στην περίπτωσή μας είναι \textbf{3}. Τιμή αναμενόμενη λόγω του depth με τιμή 2 που επιλέξαμε.
		\item \textbf{Average path length}: Είναι ο \textbf{μέσος όρος των συντομότερων μονοπατιών} για όλα τα ζεύγη κόμβων. Στο δίκτυο μας είναι \textbf{1.9760}.
		\begin{center}
			\includegraphics[width=0.4\textwidth]{photos-files/section4/graph_distance_report.JPG}
		\end{center}
	\end{itemize}
	\label{chap:basic_network_elements_4}
	
	
	\newpage
	\section{Component Measures}
	Στο δίκτυο μας, όλοι οι κόμβοι είναι συνδεδεμένοι μεταξύ τους(έμμεσα είτε άμεσα). Άρα μπορούμε να πούμε πως υπάρχει \textbf{ένα giant component}. Επομένως ο αριθμός των \textbf{weakly connected components} είναι ίσος με \textbf{1}. \par 
	Αναφορίκα με τον αριθμό των \textbf{strongly connected components}, αυτό που πρέπει να δούμε στην περίπτωση μας είναι αν υπάρχουν κανάλια-κόμβοι τα οποία δεν έχουν Featured Channels, δηλαδή δεν έχουν εξερχόμενους συνδέσμους. Έτσι μέσω του Connected Components tool απο το πεδίο Statistics του Gephi έχουμε την ακόλουθη αναφορά.
	\par
	\begin{center}
		\includegraphics[width=0.4\textwidth]{photos-files/section5/connected_components_report.JPG}
	\end{center}
	\par
	Παρατηρόντας την πιο πάνω είκονα λοιπόν, μπορούμε να επιβεβαιώσουμε τον αριθμό των weakly connected components. Όσον αφορά τον αριθμό των strongly connected components μεσω του Gephi βλέπουμε πως είναι \textbf{57}. Στο σημείο αυτο μπορούμε να εφαρμόσουμε μια διάταξη για να δούμε σχηματικά αυτους τους κόμβους ώστε να καταλάβουμε καλύτερα τι σημβαίνει. Χρησιμοποιόντας λοιπόν τον αλγόριθμο Dual Circle Layout, με Upper Order Count ίσο με 20(Πλήθος κόμβων - strong connected components + weakly connected components) με σκοπό να πάρουμε στον εξωτερικό κύκλο τα κανάλια που δεν έχουν Featured Channels(20 κανάλια, 20 διαφορετικά χρώματα). Έτσι όπως φαίνεται και πιο κάτω, στον εξωτερικό κύκλο, τα κανάλια αυτά έχουν ακμές που φτάνουν σε αυτά και κανένα δεν έχει ακμή που να ξεκινάει απο αυτά.
	\begin{center}
		\includegraphics[width=0.8\textwidth]{photos-files/section5/section5_photo1.JPG}
	\end{center}
	Να σημειωθεί οτι κρατήσαμε διαμόρφωση των κόμβων σχετκα με το μέγεθος στην σχέση seedrank χωρις αυτο να παιζει κάποιο ρόλο, γι'αυτο και ο κόμβος Samsung έχει μεγαλύτερο μέγεθος.
	\label{chap:component_measures_5}
	
	
	\newpage
	\section{Degree Measures}
	\textcolor{red}{(mikri eiagogi AN DEN FKENNEI EN OK)Στο σημέιο αυτό της analisis mas tha aaferthoume sta degree measures. ta degree measures einai...}
		
	\subsection{Maximum Degree}
	Το Maximum Degree είναι ο μέγιστος αριθμός ακμών που έχει ενας κόμβος μέσα στο δίκτυο. Στην περίπτωση που εξετάζουμε, αφορά τον κόμβο "Samsung" με τιμη 87. Αποτέλεσμα αναμενόμενο, αφού ο συγκεκριμένος κόμβος παίζει τον πιο σηαντικό ρόλο στο δίκτυο μας όπως έχουμε δεί και σε αλλες παριπτώσεις. Αυτό φαίνεται μέσω του πιο κάτω στιγμιότυπου που πηραμε απο το Gephi αφού βρήκαμε πρώτα το degree του κάθε κόμβου.
	\begin{center}
		\includegraphics[width=0.3\textwidth]{photos-files/section6/maximum_degree.JPG}
	\end{center}
	
	\subsection{Average Node Degree}
	Το Average Node Degree είναι ο μέσος αριθμός ακμών που υπάρχουν στο δίκτυο. Στο δίκτυο μας είναι ίσο με 1.961 σύμφωνα με το Degree Report που φτιάξαμε μέσω του Gephi απο το μενού Statistics.
	\begin{center}
		\includegraphics[width=0.2\textwidth]{photos-files/section6/average_degree.JPG}
	\end{center}
	
	\newpage
	\subsection{Degree Distribution} % --------------------------------------------------- Degree Distribution
	\textcolor{red}{isos na valw mia mikri isagogi?}
	
	\subsubsection{In-Degree} % ---------------------------------------------- In-Degree
	Το In-Degree είναι οι εισέρχόμενες προς κάποιον κόμβο ακμές. Στην περίπτωση μας, ο αριθμος αυτός αποτελεί τον αριθμό των καναλιών που έχουν ως Featured Channel το κανάλι που εξετάζουμε. Έτσι για κάθε κανάλι με τη βοήθεια του Gephi για το δίκτυο μας έχουμε:		
	\begin{center}
		\begin{adjustbox}{valign=t, valign=}
			\includegraphics[width=0.34\textwidth]{photos-files/section6/in-degree1.JPG}
		\end{adjustbox}
		\hfill
		\begin{adjustbox}{valign=t}
			\includegraphics[width=0.34\textwidth]{photos-files/section6/in-degree2.JPG}
		\end{adjustbox}
	\end{center}
	\newpage
	Κατανομή του In-Degree μέσω γραφικής παράστασης:
	\begin{center}
		\includegraphics[width=0.7\textwidth]{photos-files/section6/in-degree_graphical.JPG}
	\end{center}
	Μετά απο τα πιο πάνω, θα ήταν αρκετα ενδιαφέρον να δούμε πως αλλάζει το δίκτυο όσον αφορά μέγεθος και χρώμα κόμβων σε σε συνάρτηση με το In-Degree.\textcolor{red}{dipla pou thn pio katw na mpei h ipolipi ths}
	\begin{center}
		\includegraphics[width=0.6\textwidth]{photos-files/section6/in-degree_RE-layout.JPG}
	\end{center}
	Έτσι χρησιμοποιόντας τον αλγόριθμο Radial Axis Layout μπορούμε να δούμε τον διαχωρισμο που γινεται ανάμεσα στους κόμβους σε σχέση με το in-degree του κάθε καναλιού. Τα κανάλια λοιπόν χωρήστκαν σε 7 διαφορετικές ομάδες σε οριζόντιους άξονες αφου οι διαφορετικές τιμες που παρατηρούνται είναι 7 όπως είδαμε και στους πιο πάνω πίνακες. Έτσι στο σημείο αυτό μπορούμε εύκολα να δούμε τα κανάλια τα όποια υπάρχουν κατα πολυ περισσότερες φορές ως Featured channels σε άλλα. Προταγωνησικο ρόλο έχει το κανάλι της Samsung για ακόμα μια φορα ενω ακολουθούν στη συνέχια τα κανάλια SamsungUS,  KYO WON LEE κοκ.
	
	\subsubsection{Out-Degree} % -------------------------------------------- Out-Degree
	Το Out-Degree είναι οι εξέρχόμενες απο τον κάθε κόμβο ακμές. Με το δίκτυο το οποίο μελετάμε είναι ο αριθμός των Featured Channels που μπορεί να έχει ένα κανάλι όπως βλέπουμε παρακάτω.
	\begin{center}
		\begin{adjustbox}{valign=t}
			\includegraphics[width=0.34\textwidth]{photos-files/section6/out-degree1.JPG}
		\end{adjustbox}
		\hfill
		\begin{adjustbox}{valign=t}
			\includegraphics[width=0.34\textwidth]{photos-files/section6/out-degree2.JPG}
		\end{adjustbox}
	\end{center}
	\newpage
	Κατανομή του Out-Degree μέσω γραφικής παράστασης:
	\begin{center}
		\includegraphics[width=0.7\textwidth]{photos-files/section6/out-degree_graphical.JPG}
	\end{center}
	Αντίστοιχα με το In-Degree θα δούμε πως αλλάζει το δίκτυο όσον αφορά μέγεθος και χρώμα κόμβων σε συνάρτηση με το Out-Degree αυτη τη φορά.\textcolor{red}{dipla pou thn pio katw na mpei h ipolipi ths}
	\begin{center}
		\includegraphics[width=0.6\textwidth]{photos-files/section6/out-degree_RE-layout.JPG}
	\end{center}
	Με τον αντίστοιχο τρόπο που δουλέψαμε για το In-Degree προηγουμένως, δουλέψαμε και τώρα. Όπως παρατη- ρούμε, στην πρωτη θέση εξακολουθει να είναι το καναλι Samsung ενω στο προσκήνιο έχουν προστεθει αρκετα κανάλια σε σχέση με πριν. Λογικό, αφου όσο πιο πολλα Featured Channels έχει ενα κανάλι τοσο πιο εύκολα μπορεί να προσεγγίσει κοινό και να αυξήσει τις προβολές και τις εγγραφές του. Επίσης ένα παράδειγμα που πολλές φορες συμβαίνει είναι οτι με αυτον τον τρόπο ο κόσμος μπορεί να ενημερωθεί πολυ πιο γρήγορα για ενα καινούργιο προιον που έχει παρουσιαστει σε μια άλλη χωρα βλέποντας ενα προτεινόμενο κανάλι που θα προτείνει η ίδια η πλατφόρμα του YouTube μέσω των Featured Channels που έχει το κανάλι το οποίο ακολουθεί ένας χρήστης.
	
	\newpage
	\subsubsection{Total Degree} % ------------------------------------------ Out-Degree
	Το Total Degree είναι το σύνολο των ακμών που ξεκηνούν ή που καταλήγουν σε ένα κόμβο. Με άλλα λόγια, είναι ουσιαστικά το άθροισμα του In-Degree και του Out-Degree.
	\begin{center}
		\begin{adjustbox}{valign=t}
			\includegraphics[width=0.34\textwidth]{photos-files/section6/total-degree1.JPG}
		\end{adjustbox}
		\hfill
		\begin{adjustbox}{valign=t}
			\includegraphics[width=0.34\textwidth]{photos-files/section6/total-degree2.JPG}
		\end{adjustbox}
	\end{center}
	\newpage
	Κατανομή του Total Degree μέσω γραφικής παράστασης:
	\begin{center}
		\includegraphics[width=0.7\textwidth]{photos-files/section6/total-degree_graphical.JPG}
	\end{center}
	Όπως και στις δύο προηγούμενες περιπτώσεις, θα δούμε πως διαμορφώνεται το δίκτυο μας λαμβάνοντας υπόψην το Total Degree αυτή τη φορά. \textcolor{red}{dipla pou thn pio katw na mpei h ipolipi ths}
	\begin{center}
		\includegraphics[width=0.5\textwidth]{photos-files/section6/total-degree_RE-layout.png}
	\end{center}
	\par
	Το πρώτο πράγμα που μπορεί να προσέξει κανεις για το σχήμα που προέκυψε με μετρική το Total Degree είναι πως υπάρχει ο ίδιος αριθμός ομάδων κατα πλήθος κόμβων σε σχέση με πριν. Η διαφορά όμως έγγυται στο γεγονός πως όλοι σχεδόν οι κόμβοι που υπήρχαν και πριν στο Out-Degree, πέραν απο τον προφανές της Samsung, υπάρχουν και τώρα. Άρα φαίνεται πως το Out-Degree είναι αυτο που παίζει τον πιο σημαντικό ρόλο αφού όπως είπαμε και προηγουμένως είναι αυτό που καθορίζει ποια κανάλια θα προωθηθούν περισσότερο απο τον τρόπο που δουλεύει το Youtube μέσω των Featured Channels.
	\vspace{12pt}
	\par
	Τέλος να πούμε πως δεν έγινε κάποια αναφορά για το Weight Degree αφου στο δίκτυο που μελετάμε όλες οι ακμές έχουν ίσο βάρος πραγμα που δεν επηρεάζει τα δεδομένα μας. Επομένως δεν είχε νόημα η οποιαδίποτε αναφορά σε αυτό. \textcolor{red}{sioureftou full gia touto an j nmz en k dioti j sto fire etsi elalen}
	\label{chap:degree_measures_6}
	
	
	
	
	
	\newpage
	\section{Centrality measures}
	
	\subsection{Degree}
	\textcolor{red}{dame enikserw ti na grapsw, sthn ekfonisi lalei gia Degree enw sto Section6 pou en ta Degree Measures lalei gia Total Degree. enen idia touta ta 2?}
	
	\subsection{Betweenness Centrality}
	Το Betweenness Centrality δείχνει πόσο σημαντικός είναι ένας κόμβος(ως ενδιάμεσος) όταν θέλουμε να συνδέσουμε όλους τους κόμβους μεταξύ τους μέσω αυτού. Για παράδειγμα, για τον κόμβο \(n_i\) βρίσκουμε για κάθε ζεύγος κόμβων(u, w) του δικτύου τις εξής τιμές όπου και τις διαιρούμε:
	\begin{enumerate}
		\item Το σύνολο των συντομότερων μονοπατιών απο τον κόμβο \(n_i\): \( \Sigma_{u \omega}(n_i) \)
		\item Με τον αριθμο των συντομότερων διαδρομων που παιρνούν απο τον κόμβο x(τα μονοπάτια των u προς w):\( \Sigma_{u \omega} \)
	\end{enumerate}
	Αθροίζοντας το πηλίκο των διαιρέσεων των σημείων 1 και 2 βρίσκουμε το Betweenness Centrality του κόμβου x.
	Ο τύπος για την πιο πάνω διαδικάσια δίνεται απο την σχέση \(C_B(n_i)\) = $\sum$(\( \Sigma_{u \omega}(n_i) \) / \( \Sigma_{u \omega} \)).
	% https://www.youtube.com/watch?v=PuWNYB0u_gM
	
	\vspace{12pt}
	\vspace{12pt}
	Αφού καταλάβαμε πως προκύπτει το Betweenness Centrality, μπορούμε με την χρήση του Gephi να το βρούμε αυτόματα για όλους τους κόμβους μέσω των Statistics.
	\begin{center}
		\begin{adjustbox}{valign=t}
			\includegraphics[width=0.34\textwidth]{photos-files/section7/betweenness_centrality1.JPG}
		\end{adjustbox}
		\hfill
		\begin{adjustbox}{valign=t}
			\includegraphics[width=0.34\textwidth]{photos-files/section7/betweenness_centrality2.JPG}
		\end{adjustbox}
	\end{center}
	\vspace{12pt}
	Όπως φαίνεται και απο τους πιο πάνω πίνακες λοιπόν, είναι λίγες οι χώρες που έχουν μη μηδενικό Betweenness Centrality. Στην κορυφή των μετρήσεων μας είναι για ακόμη μια φορα το κανάλι της Samsung ενώ έχουν ανέβει στην κορυφή τώρα ορισμένα κανάλια όπου σε προηγούμενες μετρλησεις δεν ήταν σε τόσο υψηλή θέση. Παραδείγματα τέτοιων καναλιών είναι τα Samsung Polska(Πολωνία), Samsung Maroc, Samsung Australia, Samsung Vietnam, Samsung Osterreich(Αυστρία), Samsung Deutschland(Γερμανία), Samsung Sverige(Σουηδία), Samsung Suomi(Φινλανδία), Samsung Norge(Νορβηγία), Samsung Mexico και Samsung France. Όπως παρατηρούμε, υπάρχει μια χώρα απο κάθε ήπειρο εκτός απο την Ευρώπη που συγκεντρώνει 7 χώρες. \textcolor{red}{epia na dw an paizoun rolo oi polisis alla en nmz afou jina pou ivra en entelos diaforetika
	https://www.statista.com/statistics/237092/share-of-revenue-at-samsung-electronics-by-region/
	mporei omos na thelei na eksaplothi evropi?}
	
	\newpage
	\subsection{Closeness Centrality}
	Το Closeness Centrality είναι μια μετρηκη που αποσκοπέι στο πόσο κοντά είναι ένας κόμβος σε όλους τους άλλους. Να σημειωθεί επίσης οτι μικρότεροι αριθμοί δείχνουν πως ένας κόμβος έχει υψηλό Closeness Centrality. Τον τρόπο με τον οποίο μπορούμε να υπολογίσουμε τη μετρικλη αυτή σε ένα κατευθυνόμενο δίκτυο όπως το δικό μας μπορούμε να τον δούμε μέσω του ακόλουθου παραδείγματος.
	\begin{center}
		\includegraphics[width=0.5\textwidth]{photos-files/section7/closeness_centrality_example.JPG}
	\end{center}
	Έστω πως θέλουμε να βρούμε το Closeness Centrality για τον κόμβο C. Βρίσκουμε τον συνολικό αριθμό αριθμό κόμβων του δικτύο μας και αφαιρούμε ένα, και τον διαιρούμε με το άθροισμα των συντομότερων μονοπατιών απο τον κόμβο που εξετάζουμε προς όλους τους υπόλοιπους. Επομένως για τον κόμβο C έχουμε:
	\begin{center}
		\includegraphics[width=0.7\textwidth]{photos-files/section7/array_of_example.JPG}
	\end{center}
	Άρα το Closeness Centrality του κόμβου C είναι 7/14 = 0.5
	\vspace{12pt}
	\vspace{12pt}
	
	Επομένως με την βοήθεια του Gephi μέσω των Statistics έχουμε τις εξής τιμές για τη μετρική αυτή.
	\begin{center}
		\begin{adjustbox}{valign=t}
			\includegraphics[width=0.34\textwidth]{photos-files/section7/closeness_centrality1.JPG}
		\end{adjustbox}
		\hfill
		\begin{adjustbox}{valign=t}
			\includegraphics[width=0.34\textwidth]{photos-files/section7/closeness_centrality2.JPG}
		\end{adjustbox}
	\end{center}
	
	
	
	
	
	
	
	
	
	\subsection{Eigenvector centrality}
	
	
	\label{chap:centrality_measures_7}
	
	
	
\end{document}
